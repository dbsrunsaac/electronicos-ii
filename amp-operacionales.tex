\documentclass[stu, 12pt, letterpaper, donotrepeattitle, floatsintext, natbib, unicode]{apa7}

% PAQUETES

% Establecer el idioma en español
\usepackage[spanish]{babel}

% Gestionar las referencias bibliografícas
\usepackage{apacite}
\usepackage[utf8]{inputenc}
\usepackage{graphicx}
\usepackage{float}
\usepackage[normalem]{ulem}
\usepackage{hyperref}
\definecolor{colorEnlace}{RGB}{0, 0, 0}
\hypersetup{
	colorlinks=true,
	linkcolor=colorEnlace,
	citecolor=colorEnlace,
	urlcolor=colorEnlace,
	pdfauthor={Davis Bremdow Salazar Roa},
	pdftitle={Circuitos Lineales con Amplificadores Operacionales}
}

% Portada 
\title{Circuitos Lineales con Amplificadores Operacionales}
\author{Ruth Juana Espino Puma - 184657 \\Davis Bremdow Salazar Roa - 200353 }
\affiliation{Universidad Nacional de San Antonio Abad del Cusco}
\course{Circuitos Electrónicos II}
\professor{Jhohan Jancco Chara}
\duedate{05 de Agosto}
\begin{document}

\begin{abstract}
	
\end{abstract}

\maketitle
\newpage

% Indice
\renewcommand\contentsname{\large Indice}
\tableofcontents
\newpage

% Cuerpo
\section{Amplificadores Operacionales}
\subsection{Historia}
Inicialmente, estos dispositivos eran tubos al vacío introducidos por George Philbrick en 1948 para su uso en computadoras analógicas. Con el tiempo, Fairchild y National Semiconductor desarrollaron versiones más avanzadas como los modelos 702, 709 y 741, los cuales eran circuitos integrados de bajo costo y tamaño reducido.

El avance tecnológico permitió la incorporación de transistores JFET en la entrada de estos amplificadores, incrementando su impedancia de entrada. Modelos como el LF356 y el CA3130, con configuraciones BIFET y BIMOS respectivamente, mejoraron la velocidad y la respuesta a altas frecuencias. Además, aparecieron amplificadores operacionales específicos para diversas aplicaciones, desde amplificadores de alta corriente y voltaje hasta circuitos para comunicaciones y audio.

A pesar de la aparición de amplificadores operacionales especializados, los de propósito general siguen siendo ampliamente utilizados debido a su versatilidad y numerosas aplicaciones. La continua evolución tecnológica asegura la expansión y mejora de estos dispositivos en múltiples campos de la electrónica.

\subsection{¿Qué son los amplificadores operacionales?}



% Referencias 
\renewcommand\refname{\large\textbf{Referencias}}

\bibliographystyle{natbib}
\bibliography{bibliografia.bib}

\end{document}
